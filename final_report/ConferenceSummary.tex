\documentclass[a4paper,11pt]{report}
\usepackage[top=1in, bottom=1in, left=1in, right=1in]{geometry}
\usepackage{amsmath}

\usepackage[pdftex]{graphicx}
\usepackage[latin1]{inputenc}
\usepackage{tikz}
\usetikzlibrary{shapes,arrows}
\usepackage{caption}
\usepackage{multirow}
\usepackage{graphicx}
\usepackage[space]{grffile}
\usepackage{float}
\usepackage{multicol}
\restylefloat{table}
\restylefloat{figure}

\renewcommand{\thesection}{\arabic{section}}
\renewcommand{\labelitemi}{$\bullet$}
\everymath=\expandafter{\the\everymath\displaystyle}
\newcommand{\Lagr}{\mathcal{L}}

% define the title
\author{  Alexander Huras -- 20344660\\
  D. Scott Neil -- 20349210\\
  Myles Tan -- 20349217\\
  Riley Donelson -- 20342815\\}
\title{SYDE 462: Conference Summary
\\Group 2: Relay \\
  Adaptive Traffic Control Framework}

\begin{document}

% generates the title
%\maketitle
%\pagebreak

%\setcounter{section}{0}
%\setcounter{page}{1}
\pagenumbering{gobble}

\centerline{  \bf \LARGE Relay: Adaptive Traffic Control Framework}
\centerline{Alexander Huras -- 20344660, D. Scott Neil -- 20349210, Myles Tan -- 20349217, Riley Donelson -- 20342815}
\centerline{Department of Systems Design Engineering}


\section{Abstract}
This report aims to provide an introduction to the one-period finite state model of financial markets and discuss its relevance to linear algebra. Pertinent background information is provided to give an understanding of the problem space.
A case study is then performed that investigates how this model works for more complex portfolios.
Multiple focus assets are analysed with the created set of securities and resulting portfolios compared.
This report is concluded with a discussion of some of the limitations of this model.
\\
\begin{multicols}{2}
\section{Background Information}
The one-period finite state model assumes that there are only two dates, start (e.g. today) and end (e.g. tomorrow). The market conditions tomorrow are also assumed to be unknown. We are provided with a set of tomorrows states for a security that encompass a finite number of states for that security. Therefore, this model does not have full coverage of all possible outcomes, but still allows for sufficient analysis of potential scenarios (Cern, 2003).
\\
\indent
Within this framework, a security can be viewed as a vector of states, whose values represent the payoff at each state. For example, lets say stock X has initial price 1 (today's price) and will either increase to 3 or 2 tomorrow. Thus, it has two states ($x_1$, $x_2$) and associated payoffs (2, 1). In vector form,
\begin{align*}
	a_{stock} = \left[\begin{array}{c}2\\ 1\end{array}\right]
\end{align*}
\indent
The state of the market tomorrow determines what this stock will pay; if we are in state $x_1$, this bond would pay 1. 
It should be noted that payoffs will be negative if price goes down. This idea applies to all other security types and can incorporate additional payments such as dividends, coupon payments, etc., but for the purposes of this report it is assumed all associated payoffs are accounted for in the payoff vector. 
Also, this idea of securities as vectors can be extended to the $n$-dimensional case, in which there would be $n$ unique states. 
Continuing our example, in Figure \ref{origPayoffEx}, a fictional stock, bond, and option are plotted to give a visual representation in 2D payoff space (\textbf{vector space}).

%\begin{figure}[H]
%\centering
%\includegraphics[scale=.45]{origPayoffEx.png}
%\caption{Example Payoff Vectors}
%\label{origPayoffEx}
%\end{figure}

\indent
Next, forming a set of these payoff vectors creates a payoff matrix $A$, known as \textbf{\textit{basis assets}}. 
The size of $A$ is determined by the number of states, $n$, and number of securities, $m$. 
The security payoff vectors form a \textbf{column space}, such that each row contains the respective security payoff for a given state. 
Multiplying $A$ by an arbitrary $n$x1 vector, $x$, gives us the expected payoff in each state, $b$. Typically though, $b$ is another security, called the \textit{focus asset}, that we are trying to hedge and $x$ solves the system, creating our portfolio (Cern, 2003). 
Therefore, to solve the system, we need to find a portfolio --- \textbf{linear combination of basis assets} --- that exactly matches the payoff of our focus asset, in each state. 
This is known as a \textit{perfect hedge} --- eliminates all risk of a position in the focus asset. 
The break-even price of the focus asset can then be determined from the portfolio based on the prices of the securities in the portfolio, but this discussion is out of scope for this report. 
In our example, we will hedge our option (focus asset) with the stock and bond, which form our basis assets. In matrix form, we have:

\begin{align*}
	Ax = b \\
\end{align*}

substituting in our basis assets and focus asset

\begin{align*}
	\left[\begin{array}{cc}
		a_{bond} & a_{stock}
	\end{array}\right]x=b
\end{align*}
\begin{align*}
	\left[\begin{array}{cc}
		1 & 2\\ 
		1 & 1
	\end{array}\right]x = \left[\begin{array}{c}3\\ 4\end{array}\right]
\end{align*}

solving the system gives our portfolio

\begin{align*}
	x = \left[\begin{array}{c}5\\ -1\end{array}\right]
\end{align*}
This portfolio represents the number of units of each asset required to perfectly hedge the option; we would need to buy 5 bonds and sell 1 stock to create this hedge. Figure \ref{PortfolioEx} gives the graphical representation of this hedge in the payoff space. If we only had one of the basis assets, we would not be able to reach the payoff of our focus asset, but a linear combination allows us to create a portfolio to reach the desired payoff within the \textbf{vector space}.


Constructing the initial basis assets may create a \textbf{linearly dependent} (LD) set. 
This is found when any one of the security payoffs in $A$ can be modelled by a combination of other security payoffs in $A$.
These securities do not allow us to attain new payoffs in the market, and thus are removed from the basis assets. 
Dependent securities are eliminated by \textbf{row-reducing the security column space ($A$)} and removing any non-pivot columns. 
The resulting \textbf{linearly independent (LI) set spans the entire payoff space}, or \textit{marketed subspace}, and the market is said to be \textit{complete} (Cern, 2003). 
In our example, the marketed subspace is all of $R^2$.
The \textbf{maximum number of securities that can form a basis in the subspace is equal to the dimension of the marketed subspace} (i.e. the number of states). 
For example, a basis for a 3-state market can contain a maximum of 3 basis assets and remain linearly independent. 
If another security were added, the payoff matrix could be simplified, and made LI again, by either removing the newly added security or finding a new LI combination of the securities, if possible, forming a new basis.

\section{Case Study - Complex Portfolio}
A case study is constructed to demonstrate higher dimension payoff vectors. 
Mock portfolios will be constructed from basis assets that span the payoff space to perfectly hedge any focus assets in the same payoff space. 
This case study will attempt to investigate how a one-period finite state model analysis can be used to create a hedging portfolio that will yield a complete market. 
\\
\indent
To begin, suppose an index similar to the S\&P 500 consists of 8 securities ($bond_1$, $bond_2$, $bond_3$, $stock_4$, $stock_5$, $stock_6$, $option_7$, and $option_8$), each with 5 states --- each security has predetermined price movements for 5 different states of the market.
The goal of this case study is to find a linear combination of the different securities in the payoff matrix, forming a portfolio, to perfectly hedge the focus asset and complete the market. 
The payoff matrix consisting of all security payoff vectors can be found below:

\begin{align*}
	A=\left[\begin{array}{cccccccc}
		1 & 1 & -2 & -0.2 & -0.6 & 2 & 0.1 & -6.9\\ 
		1 & 1 & 1 & 0.1 & 0.7 & 2.1 & -0.1 & 7.1\\
		4 & 4 & -6 & -0.6 & -0.7 & 2.3 & 0.4 & -7 \\
		9 & 9 & 7 & 0.7 & 0.7 & 1.9 & 0.9 & 7 \\
		3 & 3 & -3 & -0.3 & 0.6 & 2 & -0.3 & 6.8
	\end{array}\right]
\end{align*}

 In order to obtain the basis assets we must check the linear independence (LI) of the payoff matrix. 
Finding the LI securities allows us to represent any security's payoff in our marketed subspace with a combination of our basis assets.
The goal then is to remove any redundant securities from our payoff matrix --- one's that can be represented as a combination of others. 
Putting the current payoff matrix in reduced row echelon form, a basis can be formed from the column vectors with pivots. 
Our new payoff matrix of the basis assets:
\begin{align*}
	A=\left[\begin{array}{ccccc}
		1 & -2 & -0.6 & 2 & 0.1\\ 
		1 & 1 & 0.7 & 2.1 & -0.1\\
		4 & -6 & -0.7 & 2.3 & 0.4\\
		9 & 7 & 0.7 & 1.9 & 0.9\\
		3 & -3  & 0.6 & 2 & -0.3
	\end{array}\right]
\end{align*}

concludes that our portfolio will have 5 linearly independent securities, forming the basis assets as they exist in the column space of the payoff matrix. 
More importantly, any new focus asset in $R^5$ can now be perfectly hedged by assets 1, 3, 5, 6, 7. 
The conclusion can be made that the basis assets must posses full rank in order to represent a complete market. 
\\
\indent
To demonstrate that the payoff matrix can represent any number of assets, the following matrix, $b$, with the column space of 4 different focus assets is analysed, and multiple portfolio's created.

\begin{align*}
	Ax=b=
	\left[\begin{array}{ccccc}
		4 & 4.5 & -3 & -2\\ 
		3 & 3.5 & -4.5 & 1\\
		-4 & 3.7 & 3.4 & -6\\
		8 & -4 & 2.9 & 7\\
		-6 & -3  & 2 & -3
	\end{array}\right]
\end{align*}
solving this system of equations gives
\begin{align*}
	x=
	\left[\begin{array}{ccccc}
		-0.8603 & -2.3750 & 1.5841 & 0\\
		1.7057 & -0.3261 & -0.7175 & 1\\
 		-4.2809 &   1.7402  &  0.4830 & 0\\
		2.5639  &  3.0218  & -2.7793  &  0\\
		1.9005 &  13.8013 &  -1.2704  &  0
	\end{array}\right]
\end{align*}
The portfolio matrix, $x$, is a linear combination of the basis assets that achieves the same payoff as the focus asset. More specifically, in the portfolio matrix, to achieve the focus asset in the first column of the asset matrix, $b$, we would have to:
\begin{itemize}
	\item sell 0.8603 units of $bond_1$ 
	\item buy 1.7057 units of $bond_3$
	\item sell short 4.289 shares of $stock_5$ 
	\item buy 2.5639 call options of $option_6$
	\item buy 1.9005 call options of $option_7$
\end{itemize}

Following the above purchasing plan we could perfectly hedge the asset $b_1$. 
However, it isn't possible to trade fractions of financial instruments. 
Rounding would cause inefficiency in our hedge and is a limitation of this model. 
In actual markets, it is almost impossible to find a perfect hedge. 

\section{Conclusions}
In summary, by using the one-period finite state model we can efficiently predict a portfolio that can perfectly hedge any asset in the same marketed subspace. 
Also, a given portfolio is a linear combination of all independent basis asset payoffs. 
The basis assets are assets which span the entire payoff space, that is, our marketed subspace. 
This results in the ability to replicate any focus asset in this space, thus yielding a complete market.
\\
\indent
They are known limitations that are evident in the presented case study. One limitation is the nature in which the payoff vectors are formed. These vectors represent payoffs at a certain state in the market. Each payoff is modelled based on some probabilistic distribution. If this assumed distribution is not indicative of the actual distribution than the model fails. Additionally, the model cannot account for every market state as the model itself is finite. Unrepresented states will result in the inability to produce a correct portfolio. 
\\
\indent
As discussed earlier, in most scenarios, to perfectly hedge the focus asset requires trading fractions of securities. 
This practice is not realistic and will result in inefficient hedging.
Least Squared error methods could be used to find the optimal attainable portfolio by reducing the distance, error, from the resulting portfolio vectors to the focus asset.
This would create a portfolio closest to the optimal solution.
\\
\indent
Lastly, this analysis can provide meaningful strategies for portfolio managers that are seeking to optimize returns in unforeseen market conditions by employing linear algebra concepts.
\section{References}
Cern, A. (2003). \textit{Mathematical techniques in finance: Tools for incomplete markets.} (2nd ed.). Princeton University Press.
\end{multicols}
\end{document}